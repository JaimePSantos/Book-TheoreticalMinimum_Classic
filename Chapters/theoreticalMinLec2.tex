\documentclass[../main.tex]{subfiles} 
\begin{document}

\section{Mathematial Interlude: Differential Calculus}
\begin{itemize}
    \item Definition of derivative
        \begin{equation}
            \frac{d f(t)}{d t}=\lim _{\Delta t \rightarrow 0} \frac{\Delta f}{\Delta t}=\lim _{\Delta t \rightarrow 0} \frac{f(t+\Delta t)-f(t)}{\Delta t}
        \end{equation}

    \item Sum rule
        \begin{equation}
            \frac{d(f+g)}{d t}=\frac{d(f)}{d t}+\frac{d(g)}{d t}
        \end{equation}

    \item Product rule
        \begin{equation}
            \frac{d(f g)}{d t}=f(t) \frac{d(g)}{d t}+g(t) \frac{d(f)}{d t}
        \end{equation}

    \item Chain rule
        \begin{equation}
            \frac{d f}{d t}=\frac{d f}{d g} \frac{d g}{d t}
        \end{equation}

        if f is a function of g.
    \item $\frac{d\left(t^n\right)}{d t}=n t^{n-1}$
    \item $\frac{d(\sin t)}{d t}  =\cos t$ 
    \item $\frac{d(\cos t)}{d t}  =-\sin t$ 
    \item $\frac{d\left(e^t\right)}{d t}  =e^t$
    \item $\frac{d(\log t)}{d t}  =\frac{1}{t}$

\end{itemize}

\section{Particle Motion}

\begin{itemize}
    \item The position of a particle at time $t$ can be described by
        $\vec{r}(t)$.

    \item The job of classical mechanics is to figure out $\vec{r}(t)$ from
        some initial condition and some dynamical law.

    \item The velocity is given by the rate of change of the position
        \begin{equation}
            \vec{v} = \frac{d\vec{r}}{dt} = \dot{\vec{r}}
        \end{equation}

    \item The velocity vector has magnitude $|\vec{v}|$
        \begin{equation}
            |\vec{v}|^2 = v_x^2 + v_y^2 + v_z^2
        \end{equation}

        which represents how fast the particle is moving without regard to the
        direction. This is called \textit{speed}.  

    \item Acceleration is the quantity that tells you how the velocity is
        changing. A constant velocity vector not onbly implies constant speed,
        but also a constant direction.
        \begin{equation}
            \vec{a} = \dot{\vec{v}} = \ddot{\vec{r}}
        \end{equation}

\end{itemize}

\section{Interlude 2: Integral Calculus}
\begin{itemize}

    \item Differential Calculus studies the rates of change. Integral Calculus
        has to do with sums of many tiny incremental quantities. These are
        connected.
        
    \item Consider a rectangle located at value $t$, with width $\Delta t$ and
        height $f(t)$. It follows that the area of a single rectangle $\delta
        A$ is

        \begin{equation}
            \delta A = f(t)\Delta t
        \end{equation}

    \item Adding up all the incremental areas
        \begin{equation}
            A = \sum_i f(t_i)\Delta t
        \end{equation}

    \item To get the exact answer, we take the limit in which $\Delta t$
        shrinks to zero, and the number of rectangles increases to infinity

        \begin{equation}
            A = \int_a^b f(t) dt = \lim_{\Delta t \rightarrow 0} f(t_i)\Delta t
        \end{equation}

        which is the definition of an integral between $a$ and $b$.

    \item The indefinite integral of $f(t)$ is
        \begin{equation}
            F(T) = \int_a^T f(t) dt
        \end{equation}
        where $T$ is an unknown variable which can take any value of $t$. 

    \item The \textit{fundamental theorem of calculus} states that
        \begin{equation}
            f(t) = \frac{dF(t)}{dt}
        \end{equation}

    \item The process of integration and differentiation are reciprocal: The
        derivative of the integral is the original integrand.

    \item Can we completely determine $F(t)$ knowing that its derivative is
        $f(t)$. Almost, because adding a costant to $F(t)$ doesnt change its
        derivative.

    \item Given $f(t)$, its indefinite integral is ambiguous up to a constant.

    \item An alternative way to express the fundamental theorem is
        \begin{equation}
            \int_a^b f(t) dt = F(t)|_a^b = F(b) - F(a)
        \end{equation}

        or

        \begin{equation}
            \int \frac{df}{dt}dt = f(t) + c
        \end{equation}

\end{itemize}

\section{Lecture 2 Exercises}
\subsection*{Exercise 1: Calculate the derivatives of each of these functions.
    $\\f(t)=t^4+3 t^3-12 t^2+t-6 \\  g(x)=\sin x-\cos x \\ 
\theta(\alpha)=e^\alpha+\alpha \ln \alpha \\  x(t)=\sin ^2 x-\cos x $}

\subsubsection*{a)}

\begin{itemize}
    \item \begin{equation*}\frac{d(f+g)}{d t}=\frac{d f}{d t}+\frac{d g}{d t} \end{equation*} 
    \item \begin{equation*}\frac{d f}{d t}=4 t^3+9 t^2-24 t+1\end{equation*} 
\end{itemize}

\subsubsection*{b)}

\begin{itemize}
    \item \begin{equation*} \frac{d g}{d t}=\cos x+\operatorname{sen} x\end{equation*}
\end{itemize}


\subsubsection*{c)}

\begin{itemize}
    \item \begin{equation*}\frac{d \theta}{d t}=e^\alpha+\alpha \frac{1}{\alpha}=e^\alpha\end{equation*}
\end{itemize}

\subsubsection*{d)}

\begin{itemize}
    \item \begin{equation*} \frac{d(f g)}{d t}=f(t) \frac{d g}{d t}+g(t) \frac{d t}{d t}\end{equation*} 
    \item \begin{equation*} \sin ^2 t=\sin t \cdot \sin t\end{equation*} 
    \item \begin{equation*} \frac{d \sin ^2}{d t}=\sin t \cos t+\sin t \cos t=2 \sin t \cos t \end{equation*}
    \item \begin{equation*} \frac{d x}{d t}=2 \sin t \cos t+\sin t \end{equation*}
\end{itemize}
\end{document}
