\documentclass[../main.tex]{subfiles}
\begin{document}
\section{Simple Dynamical Systems and the space of States}
\begin{itemize}
    \item The collection of all states occupied by a system is called its
        state-space. 

    \item A world whose evolution is discrete could be called stroboscopic.

    \item A system that changes with time is called a dynamic system.

    \item The dynamical law is a rule that tells us the next state given the
        current state.

    \item The variables describing a system are called its degrees of freedom.

\end{itemize}

\section{Rules that are not allowed: The Minus-First Law}
\begin{itemize}
    \item According to the principles of classical physics, a dynamic law must
        be reversible.

    \item It must obey the conservation of information, which tells you that
        every state has one arrow in and one arrow out.

\end{itemize}

\section{Interlude 1: Spaces, trigonometry, and vectors - Vectors}
\begin{itemize}
    \item Magnitude of a vector 

        \begin{equation}
            |\vec{V}| = \sqrt{V_{x}^2 + V_{y}^2 + V_{z}^2}  
        \end{equation}

    \item Multiplication of vectors

        \begin{equation}
            \vec{A}\times \vec{B} = |\vec{A}|\times |\vec{B}| \cos{\theta}
        \end{equation}

    \item If the dot product of the vectors is $0$, they are perpendicular.
\end{itemize}

\section{Interlude 1 Exercises}

\subsection*{Exercise 3: Show that the magnitude of a vector satisfies
$|\vec{A}|^2=\vec{A} \cdot \vec{A}$}

\begin{equation}
\begin{array}{l}
|\vec{A}|=\sqrt{A_{x}^{2}+A y^{2}+A_{z}^{2}} \\
|\vec{A}|^{2}=A x^{2}+A y^{2}+A z^{2} \\
\vec{A} \cdot \vec{A}=A_{x} A x+A y A y+A z A z=|\vec{A}|^{2}
\end{array}
\end{equation}

\subsection*{Exercise 4: Let $\left(A_x=2, \quad A_y=-3, A_z=1\right)$ and
$\left(B_x=-4, B_y=-3, B_z=2\right)$. Compute the mag- nitude of a $\vec{A}$
and $\vec{B}$, their dot product, and the angle between them.}

\begin{equation}
\begin{array}{l}
A_{x}=2 ; A_{y}=-3 ; A_{z}=1\\
B x=-4 ; B y=-3 ; B_{z}=2\\
|\vec{A}|=\sqrt{4+9+1}=\sqrt{14}\\
|\vec{B}|=\sqrt{16+9+4}=\sqrt{29}\\
\vec{A} \cdot \vec{B}=2 \cdot(-4)+(-3)(-3)+2=-8+9+2=3\\
|\vec{A}||\vec{B}| \cos \theta=3 \Leftrightarrow \sqrt{14} \cdot \sqrt{29} \cdot \cos \theta=3 \Leftrightarrow \cos \theta=\frac{3}{\sqrt{406}}\\
\Leftrightarrow \theta=\cos ^{-1}\left(\frac{3}{\sqrt{40^{6}}}\right) \cong 1,4.2 \mathrm{rad}
\end{array}
\end{equation}

\subsection*{Exercise 5: Determine which pair of vectors are orthogonal.
$(1,1,1)(2,-1,3)(3,1,0)(-3,0,2)$}

\begin{equation}
\begin{array}{l}
(1,1,1)(2,-1,3)=x+-1+3=4 \\
(1,1,1)(3,1,0)=3+2+0=4 \\
(1,1,1)(-3,0,2)=-3+0+2=1 \\
(2,-1,3)(3,1,0)=6-1+0=5 \\
(2,-1,3)(-3,0,2)=-6+0+6=-0 \\
(3,1,0)(-3,0,2)=-9+0+2=-7
\end{array}
\end{equation}

\subsection*{Exercise 6: Can you explain why the dot product of two vectors
that are orthogonal is 0 ?}

Because $\cos(\frac{\pi}{2}) = 0$.

\end{document}
