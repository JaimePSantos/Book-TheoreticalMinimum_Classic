\documentclass[../main.tex]{subfiles} 
\begin{document}

\section{Aristotle's Law of Motion}

\begin{itemize}
    \item The velocity of any object is proportional to the total applied
        force.

        \begin{equation}
            \vec{F} = m \vec{v}
        \end{equation}

    \item $m$ is a quantity describing the resistance of the body to being
        moved.

    \item Aristotle's law is wrong.
\end{itemize}

\section{Mass, Acceleration and Force}

\begin{itemize}

    \item Aristotle's mistake was to think that a net applied force is needed
        to keep an object moving.

    \item The right idea is that the applied force is needed to overcome the force of friction.
        
    \item An isolated object movign in free space, with no forces acting on it,
        requires nothing to keep it moving. This is the \textit{law of
        inertia}.

    \item This involves a change in the velocity of an object, and therefore an
        acceleration. 

    \item Newton's second law of motion
        \begin{equation}
            \vec{F} = m\vec{a}
        \end{equation}

\end{itemize}

\section{An Interlude on Units}
\begin{itemize}
    \item Velocity
        \begin{equation}
            \left[v\right] = \left[\frac{length}{time}\right] = \frac{m}{s}
        \end{equation}

    \item Acceleration
        \begin{equation}
            \left[a\right] = \left[\frac{length}{time}\right]\left[\frac{1}{time}\right] = \frac{m}{s^2}
        \end{equation}

    \item Force
        \begin{equation}
            \left[F\right] = \left[\frac{mass \times length}{time^2}\right] = \frac{kg \times m}{s^2} = N (Newton)
        \end{equation}

\end{itemize}

\section{Some Simple Examples of Solving Newton's Equations}
\begin{itemize}
    \item 
        \begin{equation}
            m \frac{d\vec{v}}{dt} = 0 \implies \vec{v}(t) = \vec{v}(0)
        \end{equation}

        This is newtons first law of motion: \textit{ Every object in a state
        of uniform motion tends to remain in that state of motion unless an
        external force is applied to it}.
        
    \item The first law is simply a special case of the second law when the
        force is zero.

\end{itemize}

\section{Interlude 3: Partial Differentiation - Partial Derivatives}
\begin{itemize}
    \item Assume that for every value of $x$, $y$ and $z$, there is a unique
        value $V(x,y,z)$ that varies smoothly as we vary the coordinates.

    \item Multivariable differential calculus revolves around the concept of
        \textit{partial derivatives}.

    \item If we want to change $x$ while keeping $y$ and $z$ fixed
        \begin{equation}
            \frac{dV}{dx} = \lim_{\Delta x \rightarrow 0} \frac{\Delta V}{\Delta x} = \pdv{V}{x}
        \end{equation}

        where
        \begin{equation}
            \Delta V = V (\left[x + \Delta x\right], y, z) - V(x, y, z)
        \end{equation}

    \item The second derivative is 
        \begin{equation}
            \pdv[order=2]{V}{x} = \partial_x \left(\pdv{V}{x}\right) = \partial_{x,x}V
        \end{equation}

    \item And a mixed partial derivative
        \begin{equation}
            \pdv{V}{x,y} = \partial_x \left(\pdv{V}{y}\right) = \partial_{x,y}V
        \end{equation}

        which does not depend on the order in which the derivatives are carried
        out on.

\end{itemize}

\end{document}
